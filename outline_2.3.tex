\documentclass[11pt]{article}
\usepackage[letterpaper, margin=1in]{geometry}
\usepackage{amsmath, amssymb, graphicx, epsfig, fleqn}
\setlength{\parindent}{0pt}
\newcommand{\ud}{\,\mathrm{d}}
\everymath{\displaystyle}
\def\FillInBlank{\rule{2.5in}{.01in} }
\pagestyle{empty}

\begin{document}
\begin{center}
\Large
\rm{Math 111}
\\
\rm{Chapter 2.3:  Limit Laws}
\\
\end{center}
\vspace{0.2in}
\fboxsep0.5cm

If limits exist we can do {\bf algebra with limits}.  If $\lim_{x\to a}f(x)$ and $\lim_{x\to a}g(x)$ exist, then
\begin{displaymath}
  \lim_{x\to a}f(x) + \lim_{x\to a}g(x) =  \lim_{x\to a}(f(x) + g(x))
  \end{displaymath}

\vspace{2.5in}


Other simple rules:
\begin{displaymath}
  \lim_{x\to a}x = a \quad\quad\quad   \lim_{x\to a}c = c   
  \end{displaymath}
where $c$ is a constant\\

\vspace{.5in}

(EXAMPLES)

\begin{enumerate}
  \item{
\begin{displaymath}
  \lim_{x\to 0} x^2-3 = 
  \end{displaymath}

\vspace{.5in}

}

    \vspace{1.0in}


          \item{
\begin{displaymath}
  \lim_{x\to 2} \frac{4x^2(x-2)}{x^2+x-6} = 
  \end{displaymath}

\vspace{1.5in}

  }


  \end{enumerate}

\pagebreak

(THEOREM) If $f(x)=g(x)$ when $x\neq a$ then $\lim_{x\to a}f(x)  = \lim_{x\to a}g(x) $ if the limits exist. \\

\vspace{.5in}

(EXAMPLES)
\begin{enumerate}
\item{$\lim_{h \to 0}\frac{(-6+h)^2-36}{h}$}

  \vspace{2in}
  
\item{$\lim_{t \to 0}\frac{\sqrt{9+t}-3}{t}$}

  \vspace{2in}
  
\item{$\lim_{x\to 1}\frac{x^4-1}{x^3-1}$}


  \end{enumerate}

\pagebreak



\end{document}


