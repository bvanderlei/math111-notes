\documentclass[11pt]{article}
\usepackage[letterpaper, margin=1in]{geometry}
\usepackage{amsmath, amssymb, graphicx, epsfig, fleqn}
\setlength{\parindent}{0pt}
\newcommand{\ud}{\,\mathrm{d}}
\everymath{\displaystyle}
\def\FillInBlank{\rule{2.5in}{.01in} }
\pagestyle{empty}

\begin{document}
\begin{center}
\Large
\rm{Math 111}
\\
\rm{Chapter 3.1:  Derivatives of Polynomials and Exponential Functions}
\\
\end{center}
\vspace{0.2in}
\fboxsep0.5cm

The {\bf Power Rule } for derivatives:
\begin{displaymath}
\frac{d}{dx}\left[x^n  \right] = nx^{n-1}
  \end{displaymath}
for any constant $n$

\vspace{0.5in}

  (EXAMPLES)
\begin{enumerate}
\item{$f(x) = x^8$}
  \vspace{0.2in}
\item{$p(x) = x^{100}$}
    \vspace{0.2in}
\item{$g(t) = t^{22}$}
  \end{enumerate}

Why does it work?  

\vspace{2.5in}

The {\bf derivative of a constant function}:
\begin{displaymath}
\frac{d}{dx}\left[c  \right] = 0
  \end{displaymath}
for any constant $c$

\vspace{0.1in}

(EXAMPLE)

\pagebreak

  (EXAMPLES)  Power Rule applies to \emph{any} power!

\begin{enumerate}
\item{$f(x) = \frac{1}{x^2}$}
  \vspace{0.3in}
\item{$g(x) = \frac{1}{\sqrt{x}}$}
  \vspace{0.3in}
\item{$p(s) = s^{4.6}$}
  \vspace{0.3in}
\end{enumerate}

The {\bf Constant Multiple Rule} for derivatives:
\begin{displaymath}
\frac{d}{dx}\left[cf(x)   \right] = c\frac{d}{dx}f(x)
  \end{displaymath}
if $f$ and is a differentiable functions and $c$ is a constant

\vspace{0.2in}

(EXAMPLES)\\

\begin{enumerate}
\item{$f(x) = 1.5x^{12}$}
  \vspace{0.3in}
\item{$g(x) = \frac{7}{x^4}$}
  \vspace{0.3in}
\item{$p(s) = \pi\sqrt{s}$}
  \vspace{0.3in}
\end{enumerate}


The {\bf Sum and Difference Rules} for derivatives:
\begin{displaymath}
\frac{d}{dx}\left[f(x) + g(x)  \right] = \frac{d}{dx}f(x) + \frac{d}{dx}g(x)
  \end{displaymath}
\begin{displaymath}
\frac{d}{dx}\left[f(x) - g(x)  \right] = \frac{d}{dx}f(x) - \frac{d}{dx}g(x)
  \end{displaymath}
if $f$ and $g$ are differentiable functions.

  \vspace{0.2in}

Why do these rules work?  

\pagebreak


(EXAMPLES)\\

\begin{enumerate}
\item{$f(x) = x^9 + 5x^7 - 3x^2 + 13$}
  \vspace{0.4in}
\item{$h(x) = \frac{x^2-3}{\sqrt{x}}$}
  \vspace{0.4in}
\item{$p(s) = A\sqrt{s} + \frac{B}{\sqrt[4]{s}}$}
  \vspace{0.4in}
\item{$w(y) = y^3\left(10-\frac{15}{y^4}\right)$}
    \vspace{0.4in}
\end{enumerate}

    \vspace{0.3in}

  (APPLICATIONS)
  \begin{enumerate}
  \item{Find the equation for the line tangent to $y=x^2-x^4$ at the point $(1,0)$. Find the equation for the line normal to $y=x^2-x^4$ at the point $(1,0)$.}

  \vspace{2in}

  
  \pagebreak
  
\item{Suppose $s(t) = t^3-12t$ represents the position of a moving object in cm, with time measured in seconds.
  \begin{enumerate}
  \item{Find the velocity of the object at $t=1$.}
  \item{Find the acceleration of the object at $t=1$.}
  \item{Find the distance traveled from $t=0$ to $t=\sqrt{12}$.}
  \item{At what time does the object have greater speed, $t=0$ or $t=\sqrt{6}$.}
  \end{enumerate}
}
    \end{enumerate}
  


  \vspace{3in}

The {\bf derivative of the natural exponential function}:
\begin{displaymath}
\frac{d}{dx}\left[e^x  \right] = e^x
  \end{displaymath}

\vspace{0.1in}
Why?  Because $e$ is special!
  
      \pagebreak
(EXERCISES)

    \begin{enumerate}
    \item{Find the equation of the line tangent to the curve $y=3x\sqrt{x}$ when $x=1$.}
      \vspace{3in}

      \item{Find the line that is tangent to the curve $y=e^x$ and parallel to the line $y=5x$.}
          \vspace{3in}
        \item{If $y=Ax^2 + Bx + C$, find numbers $A$, $B$, and $C$ so that $y'' + 2y = x^2 -4$.}
            \vspace{3in}
          \item{If $f(x)=|x^2-9|$, find a formula for $f'(x)$ all points where $f$ is differentiable.}
              \vspace{3in}
          \item{Find a number $c$ so that $y=c\sqrt{x}$ is tangent to $y=1.5x+6$}
              \vspace{3in}
  \end{enumerate}

\vspace{3.5in}







\end{document}


