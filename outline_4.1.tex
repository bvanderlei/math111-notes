\documentclass[11pt]{article}
\usepackage[letterpaper, margin=1in]{geometry}
\usepackage{amsmath, amssymb, graphicx, epsfig, fleqn}
\setlength{\parindent}{0pt}
\newcommand{\ud}{\,\mathrm{d}}
\everymath{\displaystyle}
\def\FillInBlank{\rule{2.5in}{.01in} }
\pagestyle{empty}

\begin{document}
\begin{center}
\Large
\rm{Math 111}
\\
\rm{Chapter 4.1 Maximum and Minimum Values}
\\
\end{center}
\vspace{0.2in}
\fboxsep0.5cm

(DEFINITIONS)

\vspace{0.1in}
Let $c$ be a number in the domain $D$ of a function $f$.
\begin{itemize}
\item{$f(c)$ is an {\bf absolute maximum} of $f$ on $D$ if $f(c)\ge f(x)$ for all $x$ in $D$.}
\item{$f(c)$ is an {\bf absolute minimum} of $f$ on $D$ if $f(c)\le f(x)$ for all $x$ in $D$.}
  \end{itemize}

\vspace{2.5in}

NOTE:  These values are sometimes called \emph{global maximum} or \emph{global minimum} or  the \emph{extreme values} of the function.

\vspace{0.1in}

(DEFINITIONS)

\vspace{0.1in}
Let $c$ be a number in the domain $D$ of a function $f$.
\begin{itemize}
\item{$f(c)$ is an {\bf local maximum} of $f$ if $f(c)\ge f(x)$ for $x$ near $c$.}
\item{$f(c)$ is an {\bf local minimum} of $f$ if $f(c)\le f(x)$ for $x$ near $c$.}
\end{itemize}


\pagebreak
(EXAMPLES)
\begin{enumerate}
\item{$f(x) = \sin{x}$ \quad $D = [0,2\pi]$}

\vspace{1.in}

\item{$g(x) = x^2$ \quad $D = (-\infty,\infty)$}

\vspace{1.in}

\item{$h(x) = x^3$ \quad $D = [-2,3]$}

\vspace{1.in}

\item{  \begin{displaymath}
p(x) =   \left\{ \begin{array}{ll}
x & 0\le x <  \frac12 \\
x-1 & \frac12 \le x \le 1\\
\end{array} \right.
  \end{displaymath}

  $D = [0,1]$
}
  \vspace{1.5in}

\item{  $q(x) = x^3-2x^2$  \quad $D=[-1,3)$.}

\end{enumerate}
    \pagebreak

    {\bf Extreme Value Theorem}:  If $f$ is continuous on a closed interval $[a,b]$, then $f$ attains an absolute maximum value
    $f(c)$ and an absolute minimum value $f(d)$ for some numbers $c$ and $d$ in $[a,b]$.

    \vspace{1.5in}
    
    (EXAMPLE) \emph{ Theorem applies:}


    \vspace{1.5in}
    
    (COUNTEREXAMPLE)  \emph{Theorem does not apply:}

    \vspace{1.5in}

    
    (COUNTEREXAMPLE)  \emph{Theorem does not apply:}

    \vspace{1.5in}

    (EXAMPLE)  \emph{Theorem does not apply:}
    \pagebreak
   
    {\bf Fermat's Theorem}:  If $f$ has a local maximum or minimum at $c$, and $f'(c)$ exists, then $f'(c)=0$.

    \vspace{2.5in}

    (EXAMPLE) \emph{ Theorem applies:}


    \vspace{1.5in}
    
    (EXAMPLE)  \emph{Theorem applies:}

    \vspace{1.5in}

    (WARNING)  \emph{Misunderstanding of Theorem:}    


    \vspace{1.5in}
    

         \pagebreak

(DEFINITION)

 A {\bf critical number} of a function $f$ is a number $c$ in the domain of $f$ such that either $f'(c)=0$ or $f'(c)$ does not exist.

 \vspace{.1in}
 
 (EXAMPLES)

 Find the critical numbers for each function

 \begin{enumerate}
 \item{$h(x) = x^4+4x^3+2$}
   \vspace{2in}
 \item{$g(x) = x^{1/3}-x^{-2/3}$}
   \vspace{2in}
 \item{$f(x) = |x^2+4x+3|$}
   \vspace{2in}
 \end{enumerate}

 
 Note:  With this new definition, {\bf Fermat's Theorem} says that if a function $f$ has a local maximum or minimum at $c$, then
 $c$ is a critical number.

\pagebreak

{\bf Closed Interval Method} for finding the extreme values of a continuous function $f$ on an interval $[a,b]$:
\begin{enumerate}
\item{Find the values of $f$ at the critical numbers of $f$ in $(a,b)$.}
\item{Find the values of $f$ at the endpoints of the interval.}
    \item{The largest value from 1. and 2. is the absolute maximum, and the smallest value is the absolute minimum.}
\end{enumerate}  

   \vspace{1in}

   
(EXAMPLES)
In each case, find the extreme values of the function on the interval.
   
 \begin{enumerate}
 \item{$h(x) = x^3-3x^2 + 1$ on the interval $[-\frac12, 4]$}
   \vspace{2in}
 \item{$f(x) =\ln{(x^2+x+1)}$ on the interval $[-1,1]$}
   \vspace{2in}
 \item{$g(x) = xe^{-x^2/8}$ on the interval $[-1, 4]$}
   \vspace{2in}
 \item{$p(x) = \frac{x}{x^2-x+1}$ on the interval $[-2, 2]$}
   \vspace{2.5in}
 \item{$q(x) = x-k\arctan{x}$ on the interval $[0,4 ]$ \quad (\emph{$k>0$ is a constant})}
   \vspace{2.5in}
   \item{$r(x) = x-2\cos{x}$ on the interval $[-2,0]$ }
   \vspace{2.5in}
   \item{$s(x) = x\sqrt{x-x^2}$ on the interval $[0,1]$ }
 \end{enumerate}

   \pagebreak


\end{document}


