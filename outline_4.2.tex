\documentclass[11pt]{article}
\usepackage[letterpaper, margin=1in]{geometry}
\usepackage{amsmath, amssymb, graphicx, epsfig, fleqn}
\setlength{\parindent}{0pt}
\newcommand{\ud}{\,\mathrm{d}}
\everymath{\displaystyle}
\def\FillInBlank{\rule{2.5in}{.01in} }
\pagestyle{empty}

\begin{document}
\begin{center}
\Large
\rm{Math 111}
\\
\rm{Chapter 4.2 Calculus Theorems}
\\
\end{center}
\vspace{0.2in}
\fboxsep0.5cm


{\bf Intermediate Value Theorem}:  If $f$ is continuous on a closed interval $[a,b]$ and $N$ is a number
between $f(a)$ and $f(b)$, then there is a number $c$ in $[a,b]$ such that $f(c)=N$.

\vspace{1.5in}


NOTES:

\begin{itemize}
\item{There may be more than one value of $c$ that satisfies the theorem.}
\vspace{1in}
\item{$f$ may take on values that are not between $f(a)$ and $f(b)$.}
  \vspace{1in}
  \end{itemize}

(EXAMPLE)

Suppose that the outside temperature this morning at 7:00 was 3$^{\circ}$C and at 11:00 the temperature was 10$^{\circ}$C.
If temperature is a continuous function of time, we can conclude by the Intermediate Value Theorem
that at some point the temperature was 8$^{\circ}$C.

In fact, we can conclude that for any temperature $N$, such that $3\leq N \leq 10$, there was some time when the temperature
took on that value.


(COUNTEREXAMPLE)
\emph{If the function is not continuous, the conclusion may or may not be true!}

\pagebreak

(APPLICATION)

Use the Intermediate Value Theorem to prove that the equation $x^3+e^x=0$ has a solution in the interval $[-1,0]$.

\vspace{1in}

One way to search for the solution in a systematic way is called the {\bf Bisection Method}.  In this method, we divide the interval
in half and apply the Intermediate Value Theorem to each subinterval.

\vspace{1.5in}

\begin{tabular}{|c| c| c| c|} \hline
$a_n$ & $b_n$ & $h_n$ & $m_n$\\ 
\hline
\hspace{1in} & \hspace{1in} & \hspace{1in} & \hspace{1in}\\ 
\hline
\hspace{1in} & \hspace{1in} & \hspace{1in} & \hspace{1in}\\ 
\hline
\hspace{1in} & \hspace{1in} & \hspace{1in} & \hspace{1in}\\ 
\hline
\hspace{1in} & \hspace{1in} & \hspace{1in} & \hspace{1in}\\ 
\hline
\hspace{1in} & \hspace{1in} & \hspace{1in} & \hspace{1in} \\
\hline
\end{tabular}



\pagebreak

{\bf Rolle's Theorem}:
Let $f$ be a function that satisfies the following conditions:

\begin{itemize}
\item{$f$ is continuous on $[a,b]$.}
\item{$f$ is differentiable on $(a,b)$.}
 \item{$f(a) = f(b)$}
  \end{itemize}

Then there exists a number $c$ in $(a,b)$ such that $f'(c)=0$.

\vspace{2.5in}

(EXAMPLE)

Let $f(x)=x^3-4x+2$ on the interval $[0,2]$.
\vspace{1in}


(APPLICATION)

Prove that there is \emph{at most} one solution to the equation $e^x+x^3 = 0$ in the interval $[-1,0]$.


\pagebreak

{\bf Mean Value Theorem}:
Let $f$ be a function that satisfies the following conditions:

\begin{itemize}
\item{$f$ is continuous on $[a,b]$.}
\item{$f$ is differentiable on $(a,b)$.}
  \end{itemize}

Then there exists a number $c$ in $(a,b)$ such that $f'(c)=\frac{f(b)-f(a)}{b-a}$.

\vspace{2in}




(EXAMPLES)
Let $f(x) = x^2$ on $[-1,3]$.


\vspace{2in}


Let $f(x) = e^{-x}$ on $[0,2]$.


\vspace{1in}

\pagebreak

(APPLICATIONS)
If $s(t)$ represents position, $s'(t)$ is velocity.  The \emph{average velocity} over a time interval $a\leq t \leq b$ is
\begin{displaymath}
v_{avg}=\frac{s(b)-s(a)}{b-a} = \frac{\Delta s}{\Delta t}
  \end{displaymath}

So if we travel in a car 280 km in 4 hrs, what can we conclude with the Mean Value Theorem?
\vspace{0.2in}
If cameras that are 5 km apart on the highway take pictures of the same car and the time between the pictures is 2.1 minutes, what
can be concluded about the speed of the car?

\vspace{1in}


Suppose that for a given function $f$, we know that $f'(x)\leq 2$ for all $x$.  If $f(4)=10$, what is largest that $f(9)$ could be?

\vspace{1.5in}

Suppose that for a given function $g$, it is known that $-1\leq f'(x) \leq 3$ for all $x$ in $[0,5]$.  What is the largest that $f(5)-f(0)$ could be?
What is the smallest that $f(5)-f(0)$ could be?

\vspace{1.5in}

Proof of Mean Value Theorem:


\pagebreak

(THEOREM)  If $f'(x)=0$ for all $x$ in $[a,b]$, then $f(x) = constant$ on $[a,b]$.

\vspace{2in}


(COROLLARY)  If $f'(x)=g'(x)$ for all $x$ in $[a,b]$, then $f(x) = g(x) + constant$ on $[a,b]$.

\end{document}


