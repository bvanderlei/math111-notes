\documentclass[11pt]{article}
\usepackage[letterpaper, margin=1in]{geometry}
\usepackage{amsmath, amssymb, graphicx, epsfig, fleqn}
\setlength{\parindent}{0pt}
\newcommand{\ud}{\,\mathrm{d}}
\everymath{\displaystyle}
\def\FillInBlank{\rule{2.5in}{.01in} }
\pagestyle{empty}

\begin{document}
\begin{center}
\Large
\rm{Math 111}
\\
\rm{Chapter 3.10:  Linear Approximation}
\\
\end{center}
\vspace{0.2in}
\fboxsep0.5cm

Idea:  The graph of a function $f$ is very close to the line tangent at a $x=a$, so long as we look \emph{near } $x=a$.

\vspace{0.1in}

Idea:  The value of the function \emph{near} $x=a$ should be \emph{near} the value of the tangent line.  In this context, we refer to
the tangent line as the {\bf linearization } of $f$ at $a$.


\vspace{3.5in}

(EXAMPLE)\\

Estimate $\sqrt{4.1}$ using the linearization of the square root function.


\vspace{3in}

Note this approach does not do very well to estimate $\sqrt{5}$, or $\sqrt{6}$.



\pagebreak

{\bf Linearization} is also called \emph{linear approximation} or \emph{tangent line approximation}.

\vspace{.15in}

(EXAMPLES)

\begin{enumerate}
  
\item{Estimate $e^{0.1}$ using a linear approximation.}

\vspace{3in}



\item{Estimate $\ln{1.2}$ using a linearization.}

\vspace{2in}



\item{Suppose $P(t)$ represents a population, and that the function satisfies the initial value problem.
\begin{displaymath}
  \left\{ \begin{array}{ll}
\frac{dP}{dt} = 2.2\sqrt{P} \\
P(0) = 900 \\
\end{array} \right.
\end{displaymath}
Use linearization to estimate $P(1)$.
}
  \end{enumerate}
\pagebreak

Another way to compute this approximations is in terms of {\bf differentials}.  The idea is that we replace $f$ with the linearization $L$
and let $dx$ be a small change in $x$.  Then $dy$ is the resulting change in $y$.

\begin{displaymath}
dy = f'(x)dx
  \end{displaymath}

\vspace{1.25in}

(EXAMPLES)

\begin{enumerate}
\item{Compute $dy$ and $\Delta y$ if $y=f(x)=x^3+x^2-2x+1$ and $x$ changes from 2 to 2.05.}


\vspace{1.75in}
\item{A sphere is measured and found to be 21 cm with a possible error in the measurement of at most 0.05 cm.  
\begin{enumerate}
\item{Estimate the maximum error made in using this measurement to compute the volume of the sphere.}
\item{Estimate the maximum \emph{relative} error made in using this measurement to compute the volume of the sphere.}
\end{enumerate}
  }
  \pagebreak
  \item{Estimate the volume of a thin cylindrical shell with radius $r$, height $h$, and thickness $dr$.}

\end{enumerate}

\vspace{3in}

A BETTER APPROXIMATION

\vspace{0.1in}

Suppose we are not satisfied with $e^{0.1}\approx 1.1$, or $\sqrt{4.1}\approx 2.025$, can we do better?
\end{document}


