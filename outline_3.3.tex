\documentclass[11pt]{article}
\usepackage[letterpaper, margin=1in]{geometry}
\usepackage{amsmath, amssymb, graphicx, epsfig, fleqn}
\setlength{\parindent}{0pt}
\newcommand{\ud}{\,\mathrm{d}}
\everymath{\displaystyle}
\def\FillInBlank{\rule{2.5in}{.01in} }
\pagestyle{empty}

\begin{document}
\begin{center}
\Large
\rm{Math 111}
\\
\rm{Chapter 3.3:  Derivatives of Trigonometric Functions}
\\
\end{center}
\vspace{0.2in}
\fboxsep0.5cm

Compare graphs of the two basic trigonometric functions $\sin{x}$ and $\cos{x}$.\\

\vspace{2.5in}


Looks like $\frac{d}{dx}[ \hspace{2cm} ] = $ \\

Check definition.  Need to know that $\sin{(x+y)} = \sin{x}\cos{y}+\sin{y}\cos{x}$
and $\lim_{h\to 0}\frac{\sin{h}}{h} = 1$.

\vspace{2.5in}

Need also to know that $\lim_{h\to 0}\frac{\cos{h}-1}{h} = 0$.

\vspace{0.1in}





\pagebreak

Looks like $\frac{d}{dx}[ \hspace{2cm} ] = $ \\


\vspace{2in}



(OTHER EXAMPLES)  For each function, calculate the derivative and then
graph the function and the derivative.



\begin{enumerate}
\item{$f(x) = \tan{x}$}
  \vspace{2in}
\item{$g(x) = \sec{x}$}
  \vspace{2in}
\item{$p(x) = \cot{x}$}
  \vspace{2in}
\end{enumerate}

\pagebreak

(APPLICATION)\\

Under certain circumstances, the equation that describes the motion of a simple
spring-mass system is given by $\frac{d^2s}{dt^2} + s = 0$.  Here $s(t)$ is a function that describes the position of the object
in units of cm, with time measured in seconds.

\vspace{4in}


\begin{enumerate}
\item{Show that $s(t) = \sin{t}$ and $s(t) = \cos{t}$ are both solutions of the equation.}
  \vspace{2in}
\item{More generally, show that $s(t) = A\sin{t} + B\cos{t}$ is a solution.}
  \vspace{2in}
\item{Find values of $A$ and $B$ if the initial postion of the object is 3 cm to the right of equilibrium,
  and the initial velocity is 0.}
    \vspace{2in}
  \item{Graph $s(t)$, $s'(t)$, and $s''(t)$.}
        \vspace{3in}
  \item{At what times is the speed of the object greatest?  Where is the object at those times?}
            \vspace{1in}
   \item{At what times is the acceleration of the object greatest?  Where is the object at those times?}

    \vspace{2in}
\end{enumerate}

\pagebreak
  (APPLICATION)
  Quantities that are periodic in nature can often be modeled with a function such as the following.
  \begin{displaymath}
P(t) = A + B\cos{\left(\frac{2\pi}{T}(t-\phi)  \right)}
      \end{displaymath}
  \vspace{3in}
  Suppose for example that $P(t)$ represents the hours of daylight in the Fraser Valley as a function of time $t$ in days.
  We suppose that $t=0$ corresponds to January 1.
\begin{enumerate}
\item{  Find values for $A$, $B$, $T$, and $\phi$ if it is assumed that the maximum
  daylight hours is 16.25, the minimum is 8.25 hours, and the longest day of the year is on June 23 ($t=174$).}
  \item{What are the number of daylight hours predicted by the model for today ($t=277$)?}
  \end{enumerate}
  

  \pagebreak
(EXERCISES)\\

\begin{enumerate}
\item{Find $y'$ if $y = \frac{t\sin{t}}{1+t}$}
  \vspace{1.5in}
\item{Find $H''\left(\frac{\pi}{3}\right)$ if $H(\theta) = \theta^2\sin{\theta}$}
  \vspace{1.5in}
\item{Find $f'(x)$ if $f(x) = x^4e^x\tan{x}$}
  \vspace{1.5in}
\item{For what values of $x$ does the graph of $g(x) = x+ 2\sin{x}$ have a horizontal tangent line?  Sketch a the graph of $g$.}
  \vspace{1.5in}

  \pagebreak
\item{Evaluate $\lim_{\theta \to 0}\frac{\sin{8\theta}}{3\theta}$}
  \vspace{1.5in}
\item{Evaluate $\lim_{t \to 0}\frac{\tan{6t}}{\sin{t}}$}
  \vspace{1.5in}
\item{Evaluate $\lim_{x \to 2}\frac{\sin{(x-2)}}{x^2-9x+14}$}
  \vspace{1.5in}
\item{Suppose again that $s(t) = A\sin{t} + B\cos{t}$ describes the position of an oscillating mass.
\begin{enumerate}
\item{Find values of $A$ and $B$ if the initial postion of the object is 2 cm to the right of equilibrium, and the initial velocity is 3 cm/s.}
  \item{What is the first time that the object passes through equilibrium?}
  \item{What is the farthest distance away from equilibrium that the object reaches? }
  \item{At what times is the object moving the fastest?}
    
\end{enumerate}
}


\end{enumerate}


\pagebreak








\end{document}


