\documentclass[11pt]{article}
\usepackage[letterpaper, margin=1in]{geometry}
\usepackage{amsmath, amssymb, graphicx, epsfig, fleqn}
\setlength{\parindent}{0pt}
\newcommand{\ud}{\,\mathrm{d}}
\everymath{\displaystyle}
\def\FillInBlank{\rule{2.5in}{.01in} }
\pagestyle{empty}

\begin{document}
\begin{center}
\Large
\rm{Math 111}
\\
\rm{Chapter 10.2:  Derivatives of Parametric Curves}
\\
\end{center}
\vspace{0.2in}
\fboxsep0.5cm

If $x$ and $y$ are functions of $t$ and we want to know how $y$ changes with respect to $x$, we need the {\bf Chain Rule}



\vspace{1in}

Again, we can find the {\bf concavity}  of the curve by finding $\frac{d^2y}{dx^2}$.

\vspace{2in}

(EXAMPLES)


\begin{enumerate}
\item{
  \begin{displaymath}
  \left\{ \begin{array}{ll}
    x = t^2 \\
    y = t^3-3t
\end{array} \right.
\end{displaymath}
}

\pagebreak

\item{
  \begin{displaymath}
  \left\{ \begin{array}{ll}
    x = 1+\sqrt{t} \\
    y = e^{t^2}\\
\end{array} \right.
  \end{displaymath}
   \vspace{0.1in}
}

  \vspace{3in}
  
  \item{
  \begin{displaymath}
  \left\{ \begin{array}{ll}
    x = t-\ln{t} \\
    y = t+\ln{t} \\
\end{array} \right.
  \end{displaymath}
     \vspace{0.1in}

}


\end{enumerate}

\end{document}
