\documentclass[11pt]{article}
\usepackage[letterpaper, margin=1in]{geometry}
\usepackage{amsmath, amssymb, graphicx, epsfig, fleqn}
\setlength{\parindent}{0pt}
\newcommand{\ud}{\,\mathrm{d}}
\everymath{\displaystyle}
\def\FillInBlank{\rule{2.5in}{.01in} }
\pagestyle{empty}

\begin{document}
\begin{center}
\Large
\rm{Math 111}
\\
\rm{Chapter 10.3: Polar Coordinates }
\\
\end{center}
\vspace{0.2in}
\fboxsep0.5cm

{\bf Polar Coordinates} are another means by which we can describe points in a plane.  Instead of giving a pair
of numbers $(x,y)$, we will give a pair of numbers $(r,\theta)$, where $r$ is the distance from the origin, and
$\theta$ is the angle between the line that connects the point with the origin and the positive $x$ axis.





\vspace{2in}


(EXAMPLES)

\vspace{2.in}

We can connect $x$ and $y$ to $r$ and $\theta$ using trigonometry.

\vspace{2in}



\pagebreak

{\bf Polar Curves} are the collection of points that satisfy an equation involving $r$ and $\theta$.
A common situation is $r = f(\theta)$, but we can also consider more general equations.\\

\vspace{.1in}
(EXAMPLES)
\begin{enumerate}
\item{$r = 3$}
\vspace{1.5in}
\item{$\theta = \pi/3$}
  \vspace{1.5in}
      \item{$r = 2\sin{\theta}$}
\vspace{2in}
      \item{$r = 1 + \sin{\theta}$}
\vspace{2in}


\pagebreak

\item{$r = \sin{3\theta}$}
  \vspace{2.5in}
  \end{enumerate}


\vspace{1in}

(TANGENT LINES)

To find the slope of a line tangent to a polar curve, we will need $\frac{dy}{dx}$ as before.

\vspace{2in}

  (EXAMPLES)
\begin{enumerate}


\item{Find the equation of the line tangent to $r =\sin{3\theta}$ at the point where $\theta=\pi/6$.}

  \pagebreak
  
\item{$r = 1 + \sin{\theta}$ at the point where $\theta=\pi/3$.}
\vspace{2.5in}
\item{$r = \theta$ at the point where $\theta=\pi$, $\theta=3\pi$, and $\theta=5\pi$.}
\vspace{3.5in}
\item{$r = \cos{2\theta}$ at the point where $\theta=\pi/4$.}
\end{enumerate}
\end{document}
