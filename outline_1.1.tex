\documentclass[11pt]{article}
\usepackage[letterpaper, margin=1in]{geometry}
\usepackage{amsmath, amssymb, graphicx, epsfig, fleqn}
\setlength{\parindent}{0pt}
\newcommand{\ud}{\,\mathrm{d}}
\everymath{\displaystyle}
\def\FillInBlank{\rule{2.5in}{.01in} }
\pagestyle{empty}

\begin{document}
\begin{center}
\Large
\rm{Math 111}
\\
\rm{Chapter 1:  Precalculus Review}
\\
\end{center}
\vspace{0.2in}
\fboxsep0.5cm

(DEFINITION)\\

A {\bf function} is:

\vspace{0.5in}

(EXAMPLES)\\

Relationships that we might think of as a function:

\vspace{0.5in}

Ways that we might describe a function:

\vspace{1.5in}

Relationships that are \emph{not} functions

\vspace{1.5in}

(NOTATION)\\

If $f(x) = \frac{x}{2}-1$ and  $g(p) = p^2+2$\\

\vspace{0.5in}

Then $f(4) = $ \hspace{3in} $g(3) = $\\

\vspace{0.5in}

$f(a) = $ \hspace{3in} $g(p+h) = $\\


\vspace{1.5in}

(DEFINITION)\\

The {\bf domain} of a function is:

\vspace{0.5in}

(DEFINITION)\\

The {\bf range} of a function is:

\vspace{0.5in}

(DEFINITION)\\

The {\bf graph} of a function is:

\vspace{0.5in}

(EXAMPLES)\\

What are the domain and range of the following functions?  Sketch the graph and label two points on the graph.

\vspace{.25in}

$y(x) = \sqrt{x-5}$

\vspace{1.5in}

$f(x) = \frac{1}{x^2-9}$

\vspace{1.5in}

$p(x) = \sqrt{4-x^2}$

%\vspace{1.5in}

%$q(x) = \frac{x^2-1}{x+1}$


\pagebreak

\begin{center}
\Large
\rm{Linear functions}
\end{center}

(DEFINITIONS)\\

A function $f$ is {\bf linear} if:

\vspace{1.0in}

{\bf Slope}

\vspace{1.5in}

(EXAMPLES)\\

Let $C$ be a temperature measured in degrees Celsius and $F$ be the same temperature measured in degrees Fahrenheit.
We know $F$ is a linear function of $C$ and that $F(0)=32$ and $F(100)=212$.  Find a formula for $F(C)$.
\begin{enumerate}
\item{Find a formula for $F(C)$.}
\item{Give an interpretation of the slope.  What are the units?}
\end{enumerate}


\vspace{1.5in}

Let $T(t)$ represent the temperature of a lake as a function of time.  $T$ is measured in degrees Celsius and $t$ is measured in hours.
Suppose that we know that $T$ is a linear function and
that $T(4) = 6$ and $T(8) = 7$.
\begin{enumerate}
\item{Find $T(t)$ and give an interpretation of the slope.  What are the units?}
\item{How much does $T$ change in 10 hours?}
\end{enumerate}


\vspace{1.5in}

For small mammals it has been determined that body mass is proportional to heart mass.  If a 4.7 kg dog has a heart mass of 33 g, find
the mass of a 1.8 kg cat.  Present your solution in terms of a linear function.

\pagebreak

\begin{center}
\Large
\rm{Piecewise defined functions}
\end{center}

In some cases it is useful to describe functions by giving different output rules depending on

(EXAMPLES)\\
  
  \begin{displaymath}
g(x) =   \left\{ \begin{array}{ll}
x^2 & x \ge 1 \\
2-x & x < 1\\
\end{array} \right.
  \end{displaymath}

  \vspace{1.5in}

    \begin{displaymath}
f(x) =   \left\{ \begin{array}{ll}
x-1 & x > 0 \\
3 & -2 \le x \le 0\\
2-x & x < -2 \\
\end{array} \right.
  \end{displaymath}

  \vspace{1.5in}

    \begin{displaymath}
  h(x) = |x|
  \end{displaymath}

    \vspace{1.5in}


  (EXAMPLES) Some situations where piecewise defined functions might make sense:



  \pagebreak

  \begin{center}
\Large
\rm{Algebra of functions}
\end{center}
If $f$ and $g$ are functions, then so are $f+g$, $f-g$, $fg$, and $\frac{f}{g}$ for the appropriate domain
\\

(EXAMPLE) \\

Suppose $a(t) = t^2 + 1$ represents the population of fish species A, $b(t) = \frac{500}{1+2t}$ represents the population of fish species B, and $p(t)$ represents the average individual mass of a fish of species A, then: \\

\begin{enumerate}
\item{$(a+b)(t) = $ \hspace{2in} and represents }
\item{$(a-b)(t) = $ \hspace{2in} and represents }
\item{$(ap)(t) = $ \hspace{2in} and represents }
  \end{enumerate}


If $f$ and $g$ are functions, then so does $f\circ g$ on the appropriate domain
\\

(EXAMPLE) \\
\begin{enumerate}
  \item{
Suppose $r(x) = 100(1-1/x)$ represents a rabbit population as a function of $x$, the amount of edible vegetation in a habitat and
$f(r) = \sqrt{r}$ represents the population of foxes as a function of the rabbit population, then: \\

$f\circ r = $ \hspace{2in} and represents
  }

    \vspace{0.5in}
    
  \item{ If $f(x) = \frac{x}{x+3}$ and $g(x) = \frac{1}{x}$, then \\
    $f(g(x)) = $  \hspace{2in} with domain \\ 

    \vspace{0.5in}
    
    $g(f(x)) = $  \hspace{2in} with domain \\
    
  }

    \vspace{0.5in}
    
  \item{ If $f(x) = \sqrt{x}$ and $g(x) = x^2+3x$, then \\
    $f(g(x)) = $  \hspace{2in} with domain \\ 

    \vspace{0.5in}
    
    $g(f(x)) = $  \hspace{2in} with domain \\
    
    }

\end{enumerate}


\pagebreak

  \begin{center}
\Large
\rm{Inverse functions}
  \end{center}
  (DEFINITION) 
  If $f$ is  a one-to-one function, then $f^{-1}$ is the function that 'undoes' $f$
  \vspace{2in}

  (EXAMPLES)

  \begin{enumerate}
  \item{
Let $C$ be a temperature measured in degrees Celsius and $F$ be the same temperature measured in degrees Fahrenheit.
Let $f$ be the function that associates $C$ to $F$.  We found earlier that $f(C) = \frac95 C + 32$.\\
Suppose now we know the temperature is 45 degrees Fahrenheit and want to compute the temperature in degrees Celsius.
We can make use of $f^{-1}$.
}
\vspace{3in}
\item{ If $g(x) = x^3$ then $g^{-1}(x) = $ \\
  
  \vspace{0.2in}
  
  If $h(w) = 3w$ then $g^{-1}(w) = $ \\
  
  \vspace{0.2in}

  If $f(y) = \frac{1}{y}$ then $f^{-1}(y) = $ \\

  \vspace{0.2in}
  
}

\item{If $g(x) = x^3+4$ then $h^{-1}(x) = $}
  
  \vspace{2in}
  
\item{If $h(x) = \frac{4x-1}{2x+3}$ then $h^{-1}(x) = $}

    \vspace{2in}
  
\item{If $f(x) = x^2-x$ for $x \ge \frac12 $, then $f^{-1}(x) = $

}
\end{enumerate}



  


\end{document}
