\documentclass[11pt]{article}
\usepackage[letterpaper, margin=1in]{geometry}
\usepackage{amsmath, amssymb, graphicx, epsfig, fleqn}
\setlength{\parindent}{0pt}
\newcommand{\ud}{\,\mathrm{d}}
\everymath{\displaystyle}
\def\FillInBlank{\rule{2.5in}{.01in} }
\pagestyle{empty}

\begin{document}
\begin{center}
\Large
\rm{Math 111}
\\
\rm{Chapter 2.1:  Rates of Change}
\\
\end{center}
\vspace{0.2in}
\fboxsep0.5cm

If a population grows by 4,360 persons over the course of 6 years, how is it changing on average? \\

\vspace{0.2in}


If a hot coffee cools by 8$^{\circ}$C in a period of 12 minutes, how is it changing on average?  \\

\vspace{0.2in}

(DEFINITION)  The {\bf average rate of change} of a function $f$ over an interval $[x_1,x_2]$ is:

\vspace{1.5in}

(EXAMPLES)
\begin{enumerate}
  \item{
What is the average rate of change of $g(x) = \sqrt{x-2}$ on the interval $[4,6]$?\\

\vspace{.5in}

What does this number represent graphically?
}
\vspace{1.5in}





\item{A falling object has position given by $f(t)= 4.9t^2$.  What is the objects {\bf average velocity} from time $t=0$ to time $t=3$?

}
  \end{enumerate}

\pagebreak

\begin{center}
\Large
\rm{Instantaneous velocity}
\end{center}

(BIG QUESTION)  How can we determine the velocity of the falling object at one particular point in time?  For example, what is the velocity at $t=3$? 
(\emph{This is sometimes called instantaneous velocity.})


\vspace{0.2in}

We might try computing average velocities for small intervals around $t=3$.

\vspace{3.5in}


If the invervals are small, we see that the numbers \emph{approach} a single value.

\vspace{.5in}

To understand why we might look at an interval $[3, 3+ \Delta t]$ and see what happens if $\Delta t $ is small.

\vspace{2.5in}

(NOTATION)


\pagebreak


\begin{center}
\Large
\rm{Tangent problem}
\end{center}

(BIG QUESTION)  How can we find the equation for a line that is tangent to curve?  \emph{Tangent means that the line touches the curve and has the same slope.}

(EXAMPLE) What is the equation for the line that is tangent to the parabola $y=x^2$ at the point $(2,4)$? \\

\vspace{1.5in}

In order to answer, we need the slope of the curve at $(2,4)$.  Let's call $(2,4)$ $P$, and lets choose another point $Q$ on the curve and find the slope of the line that joins $P$ and $Q$.

\vspace{4.5in}

We find that for $Q$ close to $P$, the values of the slope are close to: 

\vspace{.5in}

\pagebreak

To understand why, we can look at an arbitrary point $Q$.

\vspace{1.5in}

The equation for the tangent line must then be:

\vspace{1in}

\begin{center}
\Large
\rm{Derivative}
\end{center}

(BIG IDEA) For a function $f$ and a number $a$, the rate of change of the function of $f$ at $a$ is the same as the slope of the graph at $a$.

\vspace{.25in}

(BIG DEFINITION) The {\bf derivative of a function $f$ at a number $a$} is:

\vspace{3.5in}

(EXAMPLE)  If $f(x) = 2^x$, estimate the value of $f'(2)$.
\end{document}


