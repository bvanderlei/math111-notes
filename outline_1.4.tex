\documentclass[11pt]{article}
\usepackage[letterpaper, margin=1in]{geometry}
\usepackage{amsmath, amssymb, graphicx, epsfig, fleqn}
\setlength{\parindent}{0pt}
\newcommand{\ud}{\,\mathrm{d}}
\everymath{\displaystyle}
\def\FillInBlank{\rule{2.5in}{.01in} }
\pagestyle{empty}

\begin{document}
\begin{center}
\Large
\rm{Math 111}
\\
\rm{Exponential functions}
\\
\end{center}
\vspace{0.2in}
\fboxsep0.5cm

(DEFINITION)  A function $f$ is {\bf exponential} if it can be written as:

\vspace{0.5in}

(EXAMPLE) \\

$f(x) = 2^x$

\vspace{3.5in}

(NOTES) about $f(x) = a^x$

\vspace{2.0in}

(DEFINITION)  The {\bf natural exponential function} has base:

\pagebreak

\begin{center}
\Large
\rm{Logarithmic functions}
\end{center}

(DEFINITION)  The {\bf logarithmic function} of base $a$ is the \emph{inverse} of the exponential function with base $a$.

\vspace{2.5in}

(EXAMPLES) \\
\begin{enumerate}
\item{$f(x) = \log_2{x}$ 
}
  
\vspace{3.0in}

\item{$f(x) = \ln{x}$ \quad\quad ({\bf natural logarithmic function})
}
  \vspace{3.5in}
  
\item{g(x) = $\ln{(2x-1)}$

  \vspace{2.5in}

}
\end{enumerate}

  (APPLICATIONS) \\

  Suppose that $S(t) = 100e^{0.1t}$ represents the population of a growing colony of bacteria, with $t$ measured in hours.
  \begin{enumerate}
  \item{At what time is $S$ twice its starting value?}
  \item{At what time is $S$ four times its starting value?}
\end{enumerate}    

  \pagebreak
\begin{center}
\Large
\rm{Trigonometric functions}
\end{center}
(DEFINITIONS)\\

Two basic {\bf trigonometric functions}, $\sin{x}$ and $\cos{x}$, are defined by:

\vspace{3.5in}

(NOTES)


\vspace{2.5in}


(OTHER EXAMPLES)

\end{document}


