\documentclass[11pt]{article}
\usepackage[letterpaper, margin=1in]{geometry}
\usepackage{amsmath, amssymb, graphicx, epsfig, fleqn}
\setlength{\parindent}{0pt}
\newcommand{\ud}{\,\mathrm{d}}
\everymath{\displaystyle}
\def\FillInBlank{\rule{2.5in}{.01in} }
\pagestyle{empty}

\begin{document}
\begin{center}
\Large
\rm{Math 111}
\\
\rm{Chapter 2.5:  Continuity}
\\
\end{center}
\vspace{0.2in}
\fboxsep0.5cm

(DEFINITION)  A function $f$ is {\bf continuous} at $x=a$ if $\lim_{x\to a}f(x) $ exists and $\lim_{x\to a}f(x)=f(a)$
\vspace{1in}

(EXAMPLES)\\

\vspace{1in}

(DEFINITION)  A function $f$ is {\bf continuous} on an interval $[a,b]$ it is continuous for all numbers in the interval.
\vspace{1in}

(EXAMPLES)\\

\vspace{1in}


A function $f$ is {\bf not continuous} at $x=a$ if:
\begin{enumerate}
\item{}

  \vspace{0.5in}

  

\item{}

    \vspace{0.5in}

  
\item{}

    \vspace{0.5in}


  \end{enumerate}

\pagebreak

    (EXAMPLES)\\

\begin{enumerate}
\item{{\bf Jump discontinuity}
      \begin{displaymath}
g(t) =   \left\{ \begin{array}{ll}
 2-t^2 & t \leq -1 \\
e^{-t} & t > -1 \\
\end{array} \right.
  \end{displaymath}

}

  \vspace{1in}

  

\item{ {\bf Removable discontinuity}
      \begin{displaymath}
p(x) = \frac{x-4}{x^2-x-12}
  \end{displaymath}

}

    \vspace{1.5in}

  
  \item{ {\bf Infinite discontinuity}
      \begin{displaymath}
q(x) = \frac{4}{(x-3)^2}
  \end{displaymath}

          \vspace{2in}

    \begin{displaymath}
r(x) = \frac{x-4}{x^2-x-12}
  \end{displaymath}

  }

    \vspace{0.5in}


    
  \end{enumerate}


\pagebreak

(EXAMPLES)\\

\begin{enumerate}
\item{What value of $b$ makes $h$ continuous at $\pi/2$?
      \begin{displaymath}
h(t) =   \left\{ \begin{array}{ll}
 \sin{t} & t > \pi/2 \\
b-t & t \leq \pi/2 \\
\end{array} \right.
  \end{displaymath}

}

  \vspace{1in}

  \item{  Is $p$ continuous at $x=0$?  Why, or why not?
  \begin{displaymath}
p(x) =   \left\{ \begin{array}{ll}
 |x| & x \neq 0 \\
4 & x = 0\\
\end{array} \right.
  \end{displaymath}
}

    \vspace{1in}
    
  \item{Is it possible to define $w(x)$ so that $w$ is continuous at $x=1$?
  \begin{displaymath}
w(x) = \frac{x^3-1}{x-1}
  \end{displaymath}

  }

    \vspace{1in}
    
\end{enumerate}



(COMMON CONTINUOUS FUNCTIONS) \\
\vspace{0.2in}

\begin{enumerate}
\item{Power functions, $x^r$, }
  \vspace{0.1in}
\item{Exponential functions, $a^x$}
    \vspace{0.1in}
  \item{Logarithmic functions, $\log_a{x}$}
      \vspace{0.1in}
    \item{Trigonometric functions $\sin{x}$, $\cos{x}$}
        \vspace{0.1in}
\item{Absolute value function $|x|$}
        \vspace{0.1in}
  
  \end{enumerate}
    
\pagebreak

Since we can do {\bf algebra with limits}, we can also ``do  algebra `` with continuous functions.  If $f$ and $g$ are
functions which are both continuous at $x=a$, then so are 

\vspace {1.2in}

(EXAMPLES)

\vspace {1.2in}

(THEOREM) If $\lim_{x\to a}g(x)  = b $ and $f$ is continuous at $b$, then  $\lim_{x\to a}f(g(x))  = f(b) $.

\vspace{1.25in}


(EXAMPLES)
  \begin{displaymath}
\lim_{x \to 0}\cos{(x^2 + \pi)}
  \end{displaymath}




\end{document}


