\documentclass[11pt]{article}
\usepackage[letterpaper, margin=1in]{geometry}
\usepackage{amsmath, amssymb, graphicx, epsfig, fleqn}
\setlength{\parindent}{0pt}
\newcommand{\ud}{\,\mathrm{d}}
\everymath{\displaystyle}
\def\FillInBlank{\rule{2.5in}{.01in} }
\pagestyle{empty}

\begin{document}
\begin{center}
\Large
\rm{Math 111}
\\
\rm{Chapter 3.5:  Implicit Differentiation}
\\
\end{center}
\vspace{0.2in}
\fboxsep0.5cm

(PROBLEM)\\
How can we find the line tangent to the circle $x^2+ y^2 = 25$ at the point $(3,-4)$?

\vspace{0.1in}

\begin{enumerate}
\item{Idea 1}

  \vspace{2in}
\item{Idea 2}
  
  \vspace{2in}
  
\item{Idea 3}
  
  \vspace{2.5in}
  
  \end{enumerate}

\pagebreak

(EXAMPLES) 

\begin{enumerate}
  \item{Find $\frac{dy}{dx}$ if $\sqrt{x}+\sqrt{5y} = 1$}
  \vspace{2in}
\item{Find $\frac{dy}{dx}$ at the point $(1,0)$ if $xe^y=x-y$.}
  \vspace{2in}
  \item{Find $\frac{dy}{dx}$ if $\cos(xy)=1+\sin{y}$}
  \vspace{2in}

  \pagebreak
  
\item{Find the equation of the line tangent to the curve $x^3+y^3=6xy$ at the point $(3,3)$.  Sketch the curve and the tangent line.}

  \end{enumerate}

\vspace{3in}
(APPLICATION:  CHEMISTRY) \\

In simple scenarios in thermodynamics, the pressure and volume of a gas might be related with the ideal gas law  $PV = nRT$, where $T$ is
temperature, $n$ is the quantity of gas in moles, and $R$ is a constant.  Find the rate of change of volume with respect to pressure.


\vspace{1in}

In a more complex setting, the pressure and volume of a gas might be related through van der Waal's equation  $(P+\frac{n^2a}{V^2})(V-nb)=nRT$.
Here $a$ and $b$ are additional constants.  Find the rate of change of volume with respect to pressure.


\vspace{2.5in}

\pagebreak

(APPLICATION:  INVERSE FUNCTIONS) \\
If $y = f^{-1}(x)$, then $x = f(y)$.  We can find $\frac{dy}{dx}$ with implicit differentiation.

  \vspace{1.2in}
  
(EXAMPLES)
\begin{enumerate}
\item{Find $\frac{dy}{dx}$ if $y = \arcsin(x)$}

  \vspace{3.5in}

\item{Find $\frac{dy}{dx}$ if $y = \arctan(x)$.}


    \vspace{2.5in}

  
\end{enumerate}

\pagebreak

(APPLICATION:  ORTHOGONAL CURVES) \\

Using calculus, we can explain why circles ($x^2+y^2 = r^2$) are orthogonal to lines through their centers ($ax + by = 0$)


\vspace{3in}

Another example of {\bf orthogonal curves} are $y=cx^2$ and $x^2+2y^2 = k$.


\vspace{3in}

\pagebreak

(EXERCISES)

\begin{enumerate}

\item{Find $y''$ if $x^2+xy+y^2 = 3$}.

  \vspace{2in}

\item{In each case, find $\frac{dy}{dx}$ by implicit differentiation}

\begin{enumerate}
\item{$2x^2+y^2x^2-y = 9 $ }
  \vspace{2in}
\item{$xy= x^2 - \sec{y}$}
  \vspace{2in}
\item{$x-y=e^{x/y}$}
\end{enumerate}
  

  \vspace{2.5in}
\item{At what points does the curve described by $y^2=x^3+3x^2$ have horizontal tangent line? }



  \vspace{2.5in}


\item{Find the derivatives.}
\begin{enumerate}
\item{$y=(\arcsin{x})^4$ }
  \vspace{1.7in}
\item{$y=\arctan{(1+e^{5x})^2}$}
  \vspace{1.7in}
\item{$y = \frac{\arccos{t}}{t^2}$}
\end{enumerate}



\end{enumerate}
\end{document}


