\documentclass[11pt]{article}
\usepackage[letterpaper, margin=1in]{geometry}
\usepackage{amsmath, amssymb, graphicx, epsfig, fleqn}
\setlength{\parindent}{0pt}
\newcommand{\ud}{\,\mathrm{d}}
\everymath{\displaystyle}
\def\FillInBlank{\rule{2.5in}{.01in} }
\pagestyle{empty}

\begin{document}
\begin{center}
\Large
\rm{Math 111}
\\
\rm{Chapter 2.6:  Limits at Infinity}
\\
\end{center}
\vspace{0.2in}
\fboxsep0.5cm

(EXAMPLES)  Suppose the population of turtles on a remote island is modeled by the following function, where $t$ is measured in years.
\begin{displaymath}
  P(t) = \frac{15000}{50 + 250e^{-0.05t}}
  \end{displaymath}
What do we expect will happen to the size of the population after many years?\\

\vspace{2in}

(DEFINITION)  Suppose $f$ is a function defined for all $x>a$ for some constant $a$.  Then 
\begin{displaymath}
  \lim_{x\to \infty}f(x) = L
  \end{displaymath}
means $f(x)$ can be made arbitrarily close to $L$ by taking $x$ sufficiently large.  We say that $L$ is the {\bf limit} of $f$ as $x$ goes to $\infty$.

\vspace{0.5in}

(EXAMPLES)
\begin{enumerate}
  \item{
\begin{displaymath}
  \lim_{x\to \infty} \frac{x}{x-2} = 
  \end{displaymath}
}
\vspace{1.5in}


  \item{
\begin{displaymath}
  \lim_{x\to -\infty} \frac{x}{x-2} = 
  \end{displaymath}
  }

      \end{enumerate}



\pagebreak

\begin{center}
\Large
\rm{Horizontal Asymptotes}
\end{center}


(DEFINITION) The graph of a function $f$ has a {\bf horizontal asymptote} at $y=L$ if either:
 \begin{displaymath}
  \lim_{x\to \infty} f(x) = L \quad \textrm{or} \lim_{x\to -\infty} f(x) = L
  \end{displaymath}
  

\vspace{.5in}

(EXAMPLES) \\
Find equations for all vertical and horizontal asymptotes of the graphs of the following functions.
\vspace{.1in}
\begin{displaymath}
g(x) = \frac{x-3}{9x^2-2}
  \end{displaymath}

\vspace{1.7in}

\begin{displaymath}
h(x) = \frac{\sqrt{3x^2+10}}{5x-1}
  \end{displaymath}

\vspace{1.7in}

\begin{displaymath}
f(x) = \frac{5e^x}{e^x-3}
  \end{displaymath}

\vspace{1.5in}


\pagebreak


(MORE EXAMPLES)

\begin{enumerate}
  
\item{
\begin{displaymath}
  \lim_{x\to \infty} \frac{8}{e^x-2} = 
  \end{displaymath}
}

  \vspace{1.5in}

  \item{
\begin{displaymath}
  \lim_{x\to \infty} \frac{1+ x^2}{8x^2+x+1} = 
  \end{displaymath}
}


    \vspace{1.5in}
    
  \item{
\begin{displaymath}
  \lim_{x\to \infty} \ln{(x^3+1)}
  \end{displaymath}

    \vspace{1.5in}

  }


      \item{
\begin{displaymath}
  \lim_{x\to \infty} \cos{x}
  \end{displaymath}

\vspace{1in}

  }

              \item{
\begin{displaymath}
  \lim_{x\to \infty} \sin{\left(\frac{\pi}{x}\right)}
  \end{displaymath}

\vspace{1in}

  }


  \end{enumerate}

\pagebreak

If $\lim_{x\to \infty}f(x) = \infty$ and $\lim_{x\to \infty}g(x) = \infty$ there are
3 possibilities for the limit

\begin{displaymath} 
\lim_{x\to \infty}\frac{f(x)}{g(x)} 
  \end{displaymath}

\begin{enumerate}
\item{\begin{displaymath} 
\lim_{x\to \infty}\frac{f(x)}{g(x)} =
  \end{displaymath}
}
  
  \vspace{0.2in}
  
\item{\begin{displaymath} 
\lim_{x\to \infty}\frac{f(x)}{g(x)} =
  \end{displaymath}
}
  
  \vspace{0.2in}
  
\item{\begin{displaymath} 
\lim_{x\to \infty}\frac{f(x)}{g(x)} =
  \end{displaymath}
}

  \vspace{0.2in}
  
  \end{enumerate}


The outcome is determined by the \emph{rate at which the functions grow} as $x \to \infty$.

  \vspace{0.2in}
(EXAMPLES)


\end{document}


