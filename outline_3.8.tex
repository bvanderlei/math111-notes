\documentclass[11pt]{article}
\usepackage[letterpaper, margin=1in]{geometry}
\usepackage{amsmath, amssymb, graphicx, epsfig, fleqn}
\setlength{\parindent}{0pt}
\newcommand{\ud}{\,\mathrm{d}}
\everymath{\displaystyle}
\def\FillInBlank{\rule{2.5in}{.01in} }
\pagestyle{empty}

\begin{document}
\begin{center}
\Large
\rm{Math 111}
\\
\rm{Chapter 3.8:  Natural Growth/Decay}
\\
\end{center}
\vspace{0.2in}
\fboxsep0.5cm

{\bf Law of Natural Growth/Decay}\\
\begin{displaymath}
\frac{dy}{dt} = ky  
\end{displaymath} 

\vspace{1in}
This differential equation is often given together with an initial condition $y(0)=y_0$.  Together, the equation and initial condition
are known as an \emph{initial value problem}.
\begin{displaymath}
  \left\{ \begin{array}{ll}
\frac{dy}{dt} = ky \\
y(0) = y_0 \\
\end{array} \right.
\end{displaymath}

\vspace{0.2in}

(THEOREM)
The only solution to the initial value problem is $y(t) = y_0e^{kt}$.

\vspace{0.2in}

(EXAMPLE)\\
Find the solution to the initial value problem.
\begin{displaymath}
  \left\{ \begin{array}{ll}
\frac{dy}{dt} = 2y \\
y(0) = 10 \\
\end{array} \right.
\end{displaymath}

\vspace{1in}

APPLICATION:  POPULATION GROWTH

\vspace{.1in}

Suppose that a colony of bacteria grows at a rate proportional to its size (law of natural growth) and that the population
of the colony doubles in 8.5 hours.  Find the population of the colony at time $t$ if the initial population is 1200 bacteria.

\vspace{1.in}
\pagebreak

A city population is 120000 and experiences a continuous relative growth of 3\% in the population over 10 years.  What is the populaton at the
end of the 10 year period?


\vspace{2.5in}

The following statements are equivalent:

\begin{itemize}
\item{$y$ has a growth rate proportional to its size.}
\item{$y$ has a constant \emph{relative} growth rate.}
\item{$y$ grows exponentially.}
  \item{$y$ has a fixed ``doubling time''.}
  \end{itemize}

\vspace{.1in}

APPLICATIONS: EXPONENTIAL DECAY\\


\begin{enumerate}
\item{Suppose that the amount of a drug in the bloodstream decays exponentially with a half-life of 0.5 hours.  If the amount of the drug starts 
at 180 mg, how much of the drug remains 2 hours later?  How much remains 15 minutes later?  At what time $T$ will there be exactly 50 mg of the drug
in the bloodstream?}
\vspace{2in}

\pagebreak

\item{The level of carbon-14 decays exponentially and is used to date old organic objects.  The half-life of carbon-14 is known to be 5730 years.  If a  wooden object 
  contains 17.1\% of the level of carbon-14 as a living tree does, how old is the object?}

  \vspace{2.5in}

\item{A pond has water volume 200000 m$^3$.  The pond is fed by a stream at a rate of 1000 m$^3$/day, and is drained at the same rate by a stream on the opposite side so that the volume remains constant.  Suppose that at time $t=0$, 4kg of a pollutant is spilled into the pond.  Assume that the pollutant mixes uniformly around the pond and drains with the outflowing stream.  }
  \begin{enumerate}

  \item{Write down $P(t)$, the amount of pollution remaining after $t$ days.  }
  \item{How many days before the level of pollution in the pond is reduced to 0.1kg?}
    \end{enumerate}


\end{enumerate}


\pagebreak

LAW OF COOLING \\
Consider an object with temperature $T(t)$ that is exposed to an environment with contant temperature $A$.  The temperature may be
modeled with Newton's Law of Cooling.
\begin{displaymath}
  \left\{ \begin{array}{ll}
\frac{dT}{dt} = k(A-T) \\
T(0) = T_0 \\
\end{array} \right.
\end{displaymath}

\vspace{1.5in}

(EXAMPLES)
\begin{enumerate}
\item{A hot cake is removed from the oven and placed in a room of temperature 70$^{\circ}$F.  If the temperature of the cake
  follows $\frac{dT}{dt} = 0.92(70-T)$ and the inital temperature of the cake is 200$^{\circ}$F, find the temperaure at time $t=2$ hours.}
  \vspace{3in}

\pagebreak
\item{A hot coffee at 80$^{\circ}$C is left in a room at 20$^{\circ}$C.  After 10 minutes the coffee is 76$^{\circ}$C.  How long until the
coffee cools to 72$^{\circ}$C?}

\end{enumerate}



\vspace{5in}

(EXERCISES)

\begin{enumerate}
\item{It is determined that an invasive species of fish grows with a constant relative growth rate and that the population doubles in 3 years
\begin{enumerate}
\item{How long does it take to increase by a factor of 10?  }
\item{What is the value of the relative growth rate?  (Units will be fish/year/fish.)}
\end{enumerate}
}

\item{A certain radioactive material has half-life of 2.5 years. 
\begin{enumerate}
\item{How long until a mass of 300 grams of this material decays to 5 gram?}
\item{What is the decay rate at the time it reaches mass 5 gram? (Units will be grams/year.)}
\end{enumerate}
}

\item{Let $v(t)$ be the velocity of a falling object of mass 1 kg.  A simple model for the velocity is the equation $v' = 9.8 - 0.1v$.
  \\ (In this model, 9.8 is the acceleration due to gravity, and $0.1v$ is the drag force that the object experiences as it falls.)
\begin{enumerate}
\item{Find a formula for  $v(t)$ if $v(0)=0$.  (\emph{Hint:  Change variables.  Let $u(t) = 9.8-0.1v(t)$}.) }
\item{Determine what happens to the velocity as $t\to\infty$.}
\end{enumerate}

}
\end{enumerate}

\end{document}


