\documentclass[11pt]{article}
\usepackage[letterpaper, margin=1in]{geometry}
\usepackage{amsmath, amssymb, graphicx, epsfig, fleqn}
\setlength{\parindent}{0pt}
\newcommand{\ud}{\,\mathrm{d}}
\everymath{\displaystyle}
\def\FillInBlank{\rule{2.5in}{.01in} }
\pagestyle{empty}

\begin{document}
\begin{center}
\Large
\rm{Math 111}
\\
\rm{Chapter 2.2:  Limits}
\\
\end{center}
\vspace{0.2in}
\fboxsep0.5cm


(DEFINITION)  Suppose $f$ is a function defined near a number $a$.  Then 
\begin{displaymath}
  \lim_{x\to a}f(x) = L
  \end{displaymath}
means $f(x)$ can be made arbitrarily close to $L$ by taking $x$ sufficiently close to $a$.  We say that $L$ is the {\bf limit} of $f$ as $x$ goes to $a$.

\vspace{1.5in}

(EXAMPLES)
\begin{enumerate}
  \item{
\begin{displaymath}
  \lim_{x\to 0} x^2-3 = 
  \end{displaymath}

\vspace{.5in}
How close to $0$ does $x$ need to be for $x^2-3$ to be within 0.1 of the limit?

}
\vspace{1.5in}


\item{
\begin{displaymath}
  \lim_{x\to 2} 2^x = 
  \end{displaymath}

\vspace{.5in}
How close to $2$ does $x$ need to be for $2^x$ to be within 0.05 of the limit?

  }

  \pagebreak
  
  \item{
\begin{displaymath}
  \lim_{x\to 0} \frac{4x+2x^2}{x} = 
  \end{displaymath}

\vspace{1in}

  }


  \item{Suppose that
      \begin{displaymath}
g(x) =   \left\{ \begin{array}{ll}
 \frac{4x+2x^2}{x} & x \ne 0 \\
3 & x = 0\\
\end{array} \right.
  \end{displaymath}
Find $  \lim_{x\to 0} = g(x)$.

\vspace{1.3in}

  }


      \item{
\begin{displaymath}
  \lim_{x\to 2} \frac{x-2}{x^2+x-6} = 
  \end{displaymath}

\vspace{1.5in}

  }

              \item{
\begin{displaymath}
  \lim_{x\to 4} \frac{\frac14 - \frac{1}{x}}{4-x} = 
  \end{displaymath}

\vspace{.8in}

  }

                \pagebreak
                
              \item{
\begin{displaymath}
  \lim_{x\to 0} \frac{\sin{x}}{x} = 
  \end{displaymath}

\vspace{.5in}

  }


                      \item{
\begin{displaymath}
  \lim_{x\to 0} \frac{\sqrt{x^4+16}-4}{x^4} = 
  \end{displaymath}

\vspace{2.5in}

  }


  \end{enumerate}


\begin{center}
\Large
\rm{One sided limits}
\end{center}

All functions do not have limits in all circumstances.  Consider such a function:
      \begin{displaymath}
h(x) =   \left\{ \begin{array}{ll}
 0 & x < 0 \\
1 & x \ge 0\\
\end{array} \right.
  \end{displaymath}

\vspace{0.2in}

We say that $\lim_{x \to 0}h(x)$ does not exist.  

\vspace{0.2in}
(DEFINITION)
 Suppose $f$ is a function defined near a number $a$.  Then 
\begin{displaymath}
  \lim_{x\to a^+}f(x) = L
  \end{displaymath}
means $f(x)$ can be made arbitrarily close to $L$ by taking $x$ sufficiently close to $a$ with $x>a$.  We say that $L$ is the {\bf limit} of $f$ as $x$ goes to $a$ {\bf from the right}.

\vspace{2.5in}

(THEOREM) $\lim_{x\to a}f(x) = L $ if and only if $\lim_{x\to a^+}f(x) = L $ and $\lim_{x\to a^-}f(x) = L $. \\

\vspace{.5in}

(EXAMPLES)
\begin{enumerate}

\item{
  \begin{displaymath}
f(x) =   \left\{ \begin{array}{ll}
 2-2x & x \le 0 \\
\sin{x} & x > 0\\
\end{array} \right.
  \end{displaymath}
Find $\lim_{x\to 0^+}f(x)$, $\lim_{x\to 0^-}f(x)$, and $\lim_{x\to 0}f(x)$.
}

  \vspace{1.5in}
  
\item{
  \begin{displaymath}
p(x) =   \left\{ \begin{array}{ll}
 |x| & x \neq 0 \\
4 & x = 0\\
\end{array} \right.
  \end{displaymath}
Find $\lim_{x\to 0^+}p(x)$, $\lim_{x\to 0^-}p(x)$, and $\lim_{x\to 0}p(x)$.
}

  
  \vspace{1.5in}
  
  \item{
  \begin{displaymath}
g(t) =   \left\{ \begin{array}{ll}
e^t &  t < 3\\
\ln{t} & t > 3\\
\end{array} \right.
  \end{displaymath}
Find $\lim_{t\to 3^+}g(t)$, $\lim_{t\to 3^-}g(t)$, and $\lim_{t\to 0}g(t)$.
}

\end{enumerate}


\pagebreak


\begin{center}
\Large
\rm{Infinite Limits}
\end{center}

(EXAMPLE) $\lim_{x\to 0}\frac{1}{x^2} = \infty $  \\

\vspace{1in}

(DEFINITION)  Suppose $f$ is a function defined near a number $a$.  Then 
\begin{displaymath}
  \lim_{x\to a}f(x) = \infty
  \end{displaymath}
means $f(x)$ can be made arbitrarily large  by taking $x$ sufficiently close to $a$.  We say that the limit of $f$ is {\bf infinity}.

\vspace{.5in}

(EXAMPLES)

\begin{enumerate}
\item{
  \begin{displaymath}
  \lim_{x\to 0} -\frac{1}{x^2} =
  \end{displaymath}

}

  \vspace{1in}
  
\item{
  \begin{displaymath}
  \lim_{x\to 2^+} \frac{x}{x-2} =
  \end{displaymath}
}

  \vspace{1.5in}
  
\item{  \begin{displaymath}
  \lim_{x\to 2^-} \frac{x}{x-2} =
  \end{displaymath}
}
  

\end{enumerate}
\vspace{.5in}

\pagebreak

\vspace{1in}

\begin{center}
\Large
\rm{Vertical Asymptotes}
\end{center}


(DEFINITION) The graph of a function $f$ has a {\bf vertical asymptote} at $x=a$ if any of the following are true:
 \begin{displaymath}
  \lim_{x\to a} f(x) = \infty \quad\quad\quad \lim_{x\to a^+} f(x) = \infty \quad\quad\quad \lim_{x\to a^-} f(x) = \infty 
  \end{displaymath}
 \begin{displaymath}
  \lim_{x\to a} f(x) = -\infty \quad\quad\quad \lim_{x\to a^+} f(x) = -\infty \quad\quad\quad \lim_{x\to a^-} f(x) = -\infty 
  \end{displaymath}
  

\vspace{.5in}

(EXAMPLES) \\

\begin{displaymath}
\tan{x}
  \end{displaymath}

\vspace{1.5in}

\begin{displaymath}
\ln{x}
  \end{displaymath}

\vspace{1.5in}

\begin{displaymath}
f(x) = \frac{3+x}{e^x-e}
  \end{displaymath}

\vspace{1.5in}

\begin{displaymath}
g(t) =   \left\{ \begin{array}{ll}
2  &  t \le -4\\
\frac{1}{t+4} & t > -4\\
\end{array} \right.
  \end{displaymath}


\end{document}


