\documentclass[11pt]{article}
\usepackage[letterpaper, margin=1in]{geometry}
\usepackage{amsmath, amssymb, graphicx, epsfig, fleqn}
\setlength{\parindent}{0pt}
\newcommand{\ud}{\,\mathrm{d}}
\everymath{\displaystyle}
\def\FillInBlank{\rule{2.5in}{.01in} }
\pagestyle{empty}

\begin{document}
\begin{center}
\Large
\rm{Math 111}
\\
\rm{Chapter 10.1:  Parametric equations}
\\
\end{center}
\vspace{0.2in}
\fboxsep0.5cm

{\bf Goal:}  We wish to describe the path of a moving particle in 2 dimensions.  We would like to allow paths that are
curves that could not be described by the familiar $y=f(x)$.  An easy solution is to let $x$ and $y$ depend on time $t$.









\vspace{2in}



(EXAMPLES)
\begin{enumerate}
\item{
  \begin{displaymath}
  \left\{ \begin{array}{ll}
    x = t^2+3t \\
    y = t-2
\end{array} \right.
\end{displaymath}
}

\pagebreak

\item{
  \begin{displaymath}
  \left\{ \begin{array}{ll}
    x = \cos{t} \\
    y = \sin{t} \\
\end{array} \right.
  \end{displaymath}
   \vspace{0.1in}
  for $0\leq t \leq 2\pi$
}

  \vspace{3in}
  
  \item{
  \begin{displaymath}
  \left\{ \begin{array}{ll}
    x = \cos{3t} \\
    y = \sin{3t} \\
\end{array} \right.
  \end{displaymath}
     \vspace{0.1in}
  for $0\leq t \leq 2\pi$
}

\pagebreak

  \item{
  \begin{displaymath}
  \left\{ \begin{array}{ll}
    x = 1 + 6t \\
    y = -3 + 2t \\
\end{array} \right.
  \end{displaymath}
     \vspace{0.1in}
  for $0\leq t \leq 1$
}

    \vspace{3in}
    
      \item{
  \begin{displaymath}
  \left\{ \begin{array}{ll}
    x = 1 + 6t^2 \\
    y = -3 + 2t^2 \\
\end{array} \right.
  \end{displaymath}
     \vspace{0.1in}
  for $-1 \leq t \leq 1$
}

    \vspace{1in}

    \pagebreak

  \item{
  \begin{displaymath}
  \left\{ \begin{array}{ll}
    x = \frac12(e^t + e^{-t}) \\[0.15in]
    y = \frac12(e^t - e^{-t}) \\
\end{array} \right.
  \end{displaymath}
     \vspace{0.1in}
  for $-\infty \leq t \leq \infty$
}

        \vspace{3.5in}

        
      \item{
  \begin{displaymath}
  \left\{ \begin{array}{ll}
    x = t \\[0.15in]
    y = \ln{t} \\
\end{array} \right.
  \end{displaymath}
     \vspace{0.1in}
  for $0 < t \leq \infty$
}


\end{enumerate}

\pagebreak

(INTERESTING EXAMPLES)  \emph{Get help from computer to plot}
\begin{enumerate}
        \item{
  \begin{displaymath}
  \left\{ \begin{array}{ll}
    x = e^{at}\cos{t} \\
    y = e^{at}\sin{t}
\end{array} \right.
  \end{displaymath}

}

               \vspace{1in}

             \item{
  \begin{displaymath}
  \left\{ \begin{array}{ll}
    x = Ae^t\\
    y = Ae^t + Be^{-t}
\end{array} \right.
  \end{displaymath}

}

\end{enumerate}

\vspace{1in}

(APPLICATION)

A projectile is launched horizontally from the top of a 125 m tower.  If its initial velocity is 15 m/s, how where does
it land?
\end{document}
